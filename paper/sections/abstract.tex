\begin{spacing}{1.0}
    In numerous urban and political systems, the reality of patronage clashes with theoretical models that prioritize marginal voters in determining outcomes.
    This paper examines the role of patronage using novel data from Chicago's menu program, where city council members distribute public goods in their wards using a fixed budget.
    I start with the case of Alderman Bernie Stone whom citizens rumored of under-providing precincts where he lost.
    I find strong evidence of a spending shift towards his opponents from his supporters after his removal from office.
    Next, I look for a broader trend using two differences-in-differences approaches.
    I find no significant shift comparing incumbent aldermen who barely won and lost reelection.
    Another model comparing indicted aldermen to ``entrenched'' aldermen shows a notable spending shift, albeit the results are sensitive. 
    Precincts that supported an indicted alderman saw a 1.14\% decrease in spending, while those that did not saw a 2.59\% increase, amounting to a total difference of \$56,000 difference each year. 
\end{spacing}