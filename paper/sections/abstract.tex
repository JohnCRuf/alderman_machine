\begin{spacing}{1.0}
This paper investigates the political economy of conspicuous public spending. 
It defines conspicuous public spending as spending intended to signal competency to voters to win an election but does not have substantial returns to voters compared to other projects. 
A simple career-concerns model featuring a temporal tradeoff between conspicuous and inconspicuous expenditures and political experience demonstrates this effect in a rational-choice framework. 
The model has two critical implications. 
Firstly, unexpected increases in conspicuous expenditures help politicians get re-elected. 
Secondly, less-experienced politicians spend more conspicuously than their established counterparts. 
This paper uses data from Chicago's Aldermanic Menu Program to test these implications. 
The first implication is confirmed via a simple discrete-choice analysis. Differences-in-differences methods further show some evidence for the second implication. 
However, counter to the model, the Differences-in-Differences study shows an effect that decays before the next election.
Furthermore, a regression discontinuity analysis also shows an increase in spending, which is not consistent with the model.
This paper concludes by exploring alternative hypotheses that may be driving the result. 
\end{spacing}