\begin{spacing}{1.0}
    In numerous urban and political systems, the reality of machine politics and patronage clashes with theoretical political models that prioritize marginal voters to maximize vote share.
    This paper examines the role of patronage and weak electoral competition in Chicago's menu program, where city council members allocate funds for public goods in their wards.
    It focuses on the case of Alderman Bernie Stone from the 50th ward, rumored of under-providing precincts where he lost. Analysis of precinct-level spending changes after Stone's 2011 defeat provides substantial evidence of such behavior. 
    The study further explores whether this is an isolated case or a broader trend. 
    Using a differences-in-differences approach, it finds no significant difference in spending allocation among aldermen in close elections. 
    However, another analysis, grounded in the timing of indictments, reveals a notable spending shift. 
    Precincts that supported an indicted alderman saw a 1.77\% decrease in spending, while those that did not saw a 3.12\% increase, amounting to a roughly \$70,000 difference. 
    The impact of indictments on spending is statistically significant, though sensitive to the number of precincts considered and the estimation shows evidence of anticipation, both suggesting issues with the research design.
\end{spacing}