\begin{spacing}{1.0}
    In numerous urban political systems, the reality of patronage clashes with theoretical predictions that marginal voters determine policy outcomes. This paper examines the role of patronage
    using novel data from Chicago’s menu program, where city council members distribute public
    goods in their wards using a \$1.5M budget. 
    I start with the case of Alderman Bernie Stone, whom citizens alleged of under-providing precincts where he lost. 
    I find strong evidence of a spending shift towards his opponents from his supporters after his removal from office. 
    Next, I investigate if this is a broader issue using two difference-in-differences approaches. 
    I find no significant shift comparing incumbent aldermen who barely won and lost reelection. 
    Comparing indicted aldermen to “entrenched” aldermen shows a notable spending shift, albeit the results are sensitive.
    The top eight Precincts that supported an indicted alderman saw a 1.15\% spending decrease, while the top eight opposing precincts saw a 2.59\% increase, indicating a \$448,800 annual gap between the two groups.
\end{spacing}