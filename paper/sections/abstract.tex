\begin{spacing}{1.0}
    In many urban political systems, machine politics drives policy towards loyal bases, contrasting with common theories where incumbents direct policy to win marginal voters and maximize their vote share.
    This paper examines the role of patronage and weak electoral competition in allocating public goods using novel data from Chicago's menu program, where city council members distribute \$1.5M for public goods such as infrastructure maintenance and parks within their wards.
    By examining the 50th Ward's menu money spending before and after his 2011 election loss, this paper confirms the long-standing rumor of Alderman Bernie Stone favoring supportive precincts through this program.
    Using this as a baseline, it then examines whether other aldermen allocate funds preferentially using a differences-in-differences approach after the incumbent leaves office.
    We focus on two groups of aldermen where the parallel trends assumption is likely to hold: those who won by a small margin and those whose indictments forced them out of office.
    We find no statistically significant evidence for preferential spending in the former but moderate evidence in the latter.
    Precincts that supported an indicted alderman saw a 1.14\% decrease in spending, while those that did not saw a 2.59\% increase, amounting to a total difference of \$56,000 difference each year.
\end{spacing}