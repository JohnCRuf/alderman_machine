\begin{spacing}{1.0}
This paper investigates how political incentives affect the provision of public goods.
I develop a novel dataset of public goods provision in Chicago from its primary residential infrastructure program, the Aldermanic Menu Program.
This program provides each of Chicago's 50 city council members with a fixed budget for infrastructure and other ``off-menu'' projects in their wards.
This program covers many kinds of public goods, including street and sidewalk repair, street lighting, greenery, playgrounds, and public art.
This paper first investigates a well-known case where a specific alderman, Bernie Stone of the 50th ward, was repeatedly accused of using the menu program to reward his supporters and find substantial evidence that he did so, by looking at how precinct-level spending shares changed after he was defeated in 2011.
I find using a differences-in-differences design focusing on close elections to justify the parallel trends assumption that aldermen in close elections do not allocate more spending to the top quintile of supporting precincts than the bottom.
However, using another differences-in-differences design that justifies the parallel trends assumption using indictment timing, I find the top quintile precincts that supported the incumbent alderman had a 1.77\% decline in their spending fraction after the alderman was removed from office or retired due criminal charges and a 3.12\% increase for precincts that did not support the incumbent.
This corresponds to approximately a \$ 70,000 spending gap between the top and bottom quintiles of precincts. 
The null result in close elections is robust to a variety of specifications.
The result from indictments statistical significance is sensitive to the number of precincts included, while the magnitude is not.
\end{spacing}