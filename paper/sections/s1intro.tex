\section{Introduction}\label{sec:Introduction}

Do local politicians use machine-style politics to drive public goods towards their supporters? Is spending bias exacerbated or curbed by local political competition?
Because most municipalities allocate spending through collective decision-making bodies such as councils, it is usually difficult to attribute expenditures to individual politicians.
However, one such program exists in the United States: Chicago's Aldermanic Menu Program.
Through this program, the Chicago Department of Transportation (CDOT) gives each of Chicago's 50 aldermen a \$1.5 million budget, an eponymous ``menu'' of common expenditures, and a map of 311 complaints in their ward.
Using these resources, aldermen allocate funds to infrastructure maintenance, parks, and other public goods within their wards, even if they are not on the ``menu.''

This paper asks whether the menu funding towards each alderman's supporters or detractors changes after the alderman leaves office.
If so, what determines the magnitude of this change? 
To answer this, this paper uses a differences-in-differences design to compare the time trends of each supporting or opposing precinct before and after the alderman leaves office.
There are two cases where parallel trends are likely to hold: in close elections where the incumbent wins by a small margin and in cases where the incumbent is indicted and forced to leave office.
In the former, both wards where the aldermen just barely won and just barely lost should have similar spending trends.
Furthermore, both winning and losing aldermen in these cases should have similar expectations of winning, so they should behave similarly in the run-up to the election.
In the latter, the timing of the in-office indictment is uncertain, so that should preclude any anticipation.


This paper makes two contributions, firstly it publicly introduces this setting's data, and secondly it finds cases that supports \cite{dixit_londregan1996}'s theory of machine politics.
\cite{dixit_londregan1996} argue that the key to machine politics is for politicians to be more efficient at allocating public goods to their supporters than their opponents.
If this is the case, then the municipality arrives at a ``machine politics'' equilibrium where the incumbent targets all publics goods to their top supporters.
If there is no particular efficiency towards supporters, the municipality goes to a ``downsian'' equilibrium where both the incumbent and the challenger promise to allocate public goods to the most marginal voters.
The paper finds some evidence for this bifurcation in the menu program setting.
I find evidence that Bernie Stone, who was long rumored to do this, did indeed allocate more spending to his supporters than his opponents.
I find a large swing in fund allocation to the lowest and highest 20\% of precincts by campaign contributions after Stone's 2011 loss to Debra Silverstein. 
The bottom 20\% precincts saw \$20,000 more in annual public goods , while the top 20\% precincts experienced a decline nearly double of that.
I look for a broader trend using two differences-in-differences approaches, one focusing on comparing wards where the incumbent barely won reelection to those where they barely lost, and another comparing indicted aldermen to other ``entrenched'' aldermen.
I find no significant shift comparing incumbent aldermen who barely won and lost reelection.
But the indictment design shows a notable spending shift, similar in magnitude to the Stone case study.



Section~\ref{sec:background} describes the program's history, its rules, and its 2017 audit.
Section~\ref{sec:lit_review} describes the public economics literature on public goods allocation and lobbying and the spatial and urban economics literature on infrastructure allocation and how this paper relates to both literatures.
Section~\ref{sec:data_description} describes how the menu program data was collected, its summary statistics, and depicts an aggregate map of menu spending across Chicago for the past decade.
Section~\ref{sec:case_study} presents descriptive data for the case study of Alderman Bernie Stone, rumored to favor supporters with menu program funds. 
Section~\ref{sec:empframe} explains this paper's empirical strategy to determine if the Stone case study can be generalized to other aldermen.
Section~\ref{sec:results} shows the results of said empirical strategy, which finds that aldermen removed or retired due to criminal indictments, which has similar results to the Stone case study. 
However, this does not generalize to competitive elections.
The results from the indictment-based differences-in-differences design are statistically weak and sensitive to the number of precincts included, but are economically large and significant.
Section~\ref{sec:conclusions} concludes.

