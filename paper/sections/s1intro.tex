\section{Introduction}\label{sec:Introduction}

Do local politicians use machine style politics to drive public goods towards their supporters?
Is it uniform across municipalities or does it depend on other factors, such as electoral competition?
This is a hard question to answer in most contexts, as politicians rarely have unilateral control over public spending, and when they do it is often in a dictatorial or development context that does not generalize to other political systems.
Thus, due to strategic voting, it is difficult to differentiate what policies politicians prefer from what they vote for.
However, there is one such program that exists in the United States: Chicago's Aldermanic Menu Program.
This program gives each of Chicago's 50 aldermen \$1.5 million each year to spend on infrastructure maintenance and public goods within their ward, which they can spend on whatever they see fit.
They are given a ``menu'' of common expenditures, from which the program's name is derived, but can spend freely as they wish.

The precise question this paper asks is does the menu funding towards each alderman's supporters or detractors change after the alderman leaves office?
If so, what determines the magnitude of this change? 
To answer this, this paper uses a differences-in-differences design to compare the time trends of each supporting or opposing precinct before and after the alderman leaves office.
There are two cases where parallel trends are likely to hold: in close elections where the incumbent wins by a small margin and in cases where the incumbent is indicted and forced to leave office.
In the former, both wards where the aldermen just barely won and just barely lost should have similar characteristics.
Furthermore, both winning and losing aldermen in these cases should have similar expectations of winning, so they should behave similarly in the run-up to the election.
In the latter, the incumbent alderman is forced to leave office unexpectedly, so there should be no anticipatory behavior, albeit constructing a control group is more difficult.
The value added from this paper derives from its introduction of this unique setting to urban, spatial, and political economics literature, as well as its results which reinforces \cite{dixit_londregan1996}'s theory of machine politics, the first paper to explicitly do so.


Section~\ref{sec:background} of this paper describes the background of the program, including the program's history, the program's rules, and the program's 2017 audit.
Section~\ref{sec:lit_review} of this paper describes the public economics literature on public goods allocation and lobbying and the spatial and urban economics literature on infrastructure allocation and how this paper relates to both literatures.
Section~\ref{sec:data_description} of this paper describes the dataset collected for this paper. 
It first goes through the data collection process, then describes summary statistics of the data itself, and displays a map of total spending distribution.
Section~\ref{sec:case_study} presents a case study of Alderman Bernie Stone, rumored to favor supporters with menu program funds. 
This paper offers the first quantitative proof, showing changes in fund allocation to the lowest and highest 20\% of precincts by campaign contributions before and after Stone's 2011 defeat by Debra Silverstein. 
Post-defeat, the bottom 20\% precincts saw a \$20,000 yearly increase in funds, while the top 20\% precincts experienced a decrease in their budget share.
This constitutes a difference of 1.65\% of the budget, or \$21,383 per year.
Section~\ref{sec:empframe} explains this paper's empirical strategy to determine if the Stone case study can be generalized to other aldermen.
Section~\ref{sec:results} shows the results of said empirical strategy, which finds that Aldermen removed or retired due to criminal indictments, which has similar results to the Stone case study. 
However, this does not generalize to competitive elections.
The results from the indictment-based differences-in-differences design are statistically weak and sensitive to the number of precincts included, but are economically large and significant.
Section~\ref{sec:conclusions} concludes.

