\section*{Conclusions}

Overall we find some evidence that aldermen may sometimes disproportionately allocate spending to their most supporting precincts, particularly when they are long-entrenched and not facing a competitive election.
The paper starts by verifying a rumor that a particular alderman disproportionately allocated spending to his most supporting precincts.
Then it uses two applications of a differences-in-differences research design to arrive at this fact.
The first application focuses on aldermen who lost by a small margin, and finds that the evidence for disproportionate spending is very weak.
While the magnitude of the estimated effect on the top precincts in the close election design is large, it is not even close to statistically significant, and even the sign of the effect is not robust to changes in the number of supporting/opposing precincts included in the sample.
We find that the effect is economically large and statistically significant when we use the indictment design.
The statistical significance is sensitive to parameters such as the number of supporting or opposing precincts included per ward, but the economic significance is not and stays largely the same for this design so long as you decrease the number of precincts.
This is likely due to the fact that increasing the number of precincts dilutes the average treatment effect, so it decreases the aggregated treatment effect.

The results help build on the burgeoning urban economics of infrastructure literature by showing that political incentives can distort the allocation of infrastructure spending.
Secondly, the differences between the competitive election and indictment designs show that the electoral competition can be a powerful force in constraining the clientelistic tendencies of politicians.
There is also a lesson in urban planners that while discretion can be useful, it can also be abused and lead to unintended consequences.
Therefore, the capacity for discretion should be carefully considered when designing a program.
