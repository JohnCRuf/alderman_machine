\section{Literature Review}\label{sec:lit_review}

Most literature on political allocation of public goods in political economy focuses intently on theoretical models to explain behavior but comparatively little empirical testing of that behavior. 
This literature looks primarily at the political incentives to redistribute.
The early literature in political economy focused on variations of targetting the median voter, the seminal paper being \cite{downs1957economic}.
This earlier literature was largely successful in explaining platform convergence in two-party systems.
However, many political scientists and historians found this model wanting in addressing obvious cases of machine politics \cite{rakove1975don} \cite{golway2014machine}.
The harsh examples of the New York City Tammany Hall machine and the Chicago Democratic Machine are puzzling in a downsian framework, as they maintained power through tight control of patronage jobs and distributing public goods to reward supporters.
The resolution to this puzzle came in \cite{dixit_londregan1996}, which develops a general model of politics that encompasses both the median voter and patronage models.
The model they use matches the Chicago setting to a tee: they assume that politicians have a fixed budget to allocate to public goods and can distribute it to pressure groups to maximize their vote share.
In that model they find two equilibria: one where politicians allocate to supporters and one to marginal voters.
The critical difference between the two is whether or not the politician's spending is more effective on supporters. 
If a politician has specific ties that allow them to distribute more benefits at a lower cost to supporters, they will allocate to supporters.
This difference drives our choice of estimator, as this rules out some traditional continuous-treatment tools. 

In addition, this paper is related to the lobbying literature, which studies how special interest groups influence policy. 
The key paper in this literature is \cite{grossman_helpman_1994}, which develops a model of lobbying where firms can lobby politicians to influence policy that is determined by seats in a legislature. 
Their model finds that the party expected to win will cater more towards special interests.

Numerous empirical papers look at the political allocation of public goods.
For example, \cite{finan2021electoral}'s study looks at the allocation of funds from Brazil's federal legislature, where similarly, each of the 513 legislators in Brazil receives a fixed budget of BRL\$1.5 million each year for similar public infrastructure projects.  
Furthermore, electoral competition in Brazil is intense: Only 75\% of legislators even choose to run for reelection compared to Chicago's city council's actual reelection rate of 87\%.
Furthermore, incumbents can be challenged by other incumbents due to overlapping districts, whereas in Chicago, this problem only rarely happens when wards are redrawn.
In this paper, Finan and Mazzocco estimate a structural model to find that 26\% of public funds are distorted relative to a social planner's allocation. 
After estimating their structural model, they also found that implementing an approval voting system would reduce the distortions by 7.5\%. 
They also find that term limits may reduce distortion but increase corruption.
In addition, there is \cite{frank_hoopes_lester_2022} find that governors in the US select place-based tax incentive locations more often when the tract's state representative is a member of the governor's party and is greatest with Republican governors.
However, this paper borrows the most from the \cite{fowleretalquidproquo} study, which uses a combination of regression discontinuity and first-differences design to study whether there is evidence of successful corporate campaign contributions influencing the stock prices of the donors. 
From this two-pronged approach, Fowler et al. found that there really is no impact of a preferred candidate on winning, and thus, it is hard to argue that campaign contributions are a profitable venture for companies. 

This study is different from traditional political economy studies as it focuses on a municipal environment where the public good at play is primarily infrastructure. 
Thus, it is also related to the literature on municipal infrastructure provision and the new quantitative spatial economics literature.
This includes \cite{Glaeser2018political}, \cite{Fajgelbaum2023}, \cite{treb_arkolakis_2022_infrastructure}, and \cite{bordeu2023commuting}.
Glaeser's seminal paper focuses on infrastructure's ``visible'' and ``invisible'' effects. 
In particular, he finds that governments spend too much on new infrastructure projects and not enough on maintenance.
Furthermore, local voters are less likely to support new projects due to noise, land use, and other externalities from new construction.
Glaeser uses this framework to explain the decline of urban mega-projects.
Our framework differs as we look at the allocation of relatively low-nuisance maintenance and public goods projects rather than new construction.
Thus, we see the opposite problem in maintenance: electoral concerns lead to too much spending on supporters and not enough on the rest of the municipality.
Fajgelbaum et al. examine how political economy influenced the planning of California's high-speed rail (CHSR) project.
They find that preferences for widespread approval lead to the planner placing CHSR stations farther from dense metro areas than a politically blind planner.
Treb and Arkolakis' paper uses a quantitative spatial model to evaluate the impact of improving any segment of the infrastructure network on the entire network's welfare and finds in an empirical application that there are highly variable returns to investment across different links in the network.
Finally, Bordeu looks at how infrastructure is allocated across a similarly decentralized city, Santiago, Chile, and finds that the sub-city municipalities over-invest in core areas and under-invest in areas near their boundary using a quantitative spatial model.
This misallocation results in higher-cross-jurisdiction commuting costs, less concentrated employment, and a more comprehensive spatial distribution of production.
She finds that infrastructure centralization would increase aggregate infrastructure investment and population and yield large welfare gains.