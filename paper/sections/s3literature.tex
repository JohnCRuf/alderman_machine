\section{Literature Review}\label{sec:lit_review}

Most literature on political allocation of public goods focuses on theoretical models to explain behavior but comparatively little empirical testing of that behavior. 
This literature looks primarily at the political incentives to redistribute.
The early literature in political economy focused on variations targeting the median voter, the seminal paper being \cite{downs1957economic}.
This earlier literature was largely successful in explaining platform convergence in two-party systems.
However, many political scientists and historians found this model lacking in describing cases of machine politics \citep{rakove1975don, golway2014machine}.
The harsh examples of the New York City Tammany Hall machine and the Chicago Democratic Machine are puzzling in a Downsian framework, as they maintained power through tight control of patronage jobs and distributing public goods to reward supporters.
The resolution to this puzzle came in \cite{dixit_londregan1996}, which developed a general political model encompassing both patronage and targeting median voters.
The model they use matches the Chicago setting to a tee: they assume that electorally motivated politicians have a fixed budget to allocate and can distribute it to a fixed number of pressure groups.
In that model, they find two equilibria: one where politicians give to supporters and one to marginal voters, and the equilibrium depends on the incumbent party being more efficient at distributing to supporters.


Since early contributions to the political economy literature \citep{grossman_helpman_1996}, much work still focuses on identifying the effect of special interest groups on policy outcomes \citep{bombardini_trebbi2020}. 
This literature has taken different methodological approaches to quantifying this influence.
One approach is to use structural models to simulate outcomes.
This previous literature finds that influencing politics has high returns but that political influence has relatively small effects on policy outcomes on a national level and moderate effects at a local one \citep{kang2015,finan2021electoral}.
Another approach is to take a more reduced-form approach and find some form of shock to identify how various outcomes change.
For example, \cite{frank_hoopes_lester_2022} find that governors in the US select place-based tax incentive locations more often when the tract's state representative is a member of the governor's party and is greatest with Republican governors.
In another paper, \cite{fowleretalquidproquo} use an election regression discontinuity design and a first-differences design exploiting shifts in betting market beliefs and find that having a firm’s supported candidate win has little influence on the firm’s stock price in either framework.

This study is different from traditional political economy studies as it focuses on a municipal environment where the public good at play is primarily infrastructure. 
Thus, it is also related to the literature on municipal infrastructure provision and the new quantitative spatial economics literature.
This literature includes \cite{Glaeser2018political}, \cite{Fajgelbaum2023}, and \cite{bordeu2023commuting}.
Glaeser's seminal paper focuses on infrastructure's ``visible'' and ``invisible'' costs.
Overall, voters prefer overspending, as they see the benefits of infrastructure but not the costs.
Local voters prefer under-building, as they see the externalities of the infrastructure.
Glaeser uses this framework to explain why urban mega-projects declined in the US as urban voters became more organized.
\cite{Fajgelbaum2023} examine how political economy influenced the planning of California's high-speed rail (CHSR) project.
They find that preferences for widespread approval lead to the planner placing CHSR stations farther from dense metro areas than a politically blind planner.
Finally, \cite{bordeu2023commuting} looks at how infrastructure is allocated across a similarly decentralized city, Santiago, Chile, and finds that the sub-city municipalities over-invest in core areas and under-invest in areas near their boundary using a quantitative spatial model.
She finds that infrastructure centralization would increase aggregate infrastructure investment  and yield large welfare gains.