\section{Conclusions}\label{sec:conclusions}

Overall, this paper finds some evidence that aldermen may sometimes disproportionately allocate spending to their most supporting precincts, particularly when they are long-entrenched and not facing a competitive election.
The paper starts by verifying an allegation that a particular alderman disproportionately allocated spending to his most supporting precincts.
Then, it uses two applications of a difference-in-differences research design to arrive at this fact.
The first application focuses on aldermen who lost by a small margin and finds that the evidence for disproportionate spending is very weak.
While the magnitude of the estimated effect for the top supported precincts in the close election design is large, it is not even close to statistically significant.
However, The effect is economically large and statistically significant in the indictment design.
The statistical significance is sensitive to parameters such as the number of supporting or opposing precincts per ward.
Still, this design's economic significance stays the same regardless of the number of
precincts included.

The results help build on the burgeoning urban economics of infrastructure literature by showing that political incentives can distort the allocation of infrastructure spending.
Secondly, the differences between the competitive election and indictment designs show that electoral competition can be a powerful force in constraining the clientelistic tendencies of politicians.
There is also a lesson in urban planners that while discretion can be useful, it can also be abused and lead to unintended consequences.
Therefore, the capacity for discretion should be carefully considered when designing a program.
