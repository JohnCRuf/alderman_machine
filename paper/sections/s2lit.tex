\section*{Literature Review}
Academics have written many papers on the usage of public funds by elected officials, but there are relatively few cases where an elected official gets direct, personal control over budget allocations. 
The cases in the literature are much more common, where politicians have indirect control over public spending via voting in governing bodies
An example of this kind of study is \cite{levitt1997impact}, where Levitt found that US congresspeople who can get more funds allocated to their constituents tend to win more votes. 
Levitt estimated this via an IV approach that exploited expenditures to areas outside the congressperson's district but within the same state to disentangle the reverse causality between reelection probabilities and political allocations. 

Another study that attempts to identify pandering is \cite{BundrickPandering}'s 2021 paper on political pandering in Economic Development Incentives (EDIs), which used a similar methodology to determine if EDI placement causes voters to vote for the incumbent party more often in gubernatorial elections. 
The study concludes with a two-way-fixed effects analysis that voters are unresponsive to EDIs and that public officials "do not allocate EDIs based on previous election outcomes." 

Then there is \cite{finan2021electoral}'s study that looks at the allocation of funds from Brazil's federal legislature, where similarly, each of the 513 legislators in Brazil receives a fixed budget of BRL\$1.5 million each year for similar public infrastructure projects.  
Furthermore, electoral competition in Brazil is intense: Only 75\% of legislators choose to run for reelection compared to Chicago's city council reelection rate of 87\% according to the data I have gathered. 
Furthermore, incumbents can be challenged by other incumbents due to overlapping districts, whereas in Chicago, this problem only rarely happens in the case of significant redistricting. 
In this paper, Finan and Mazzocco find that 26\% of public funds are distorted relative to the allocation a social planner would give. 
After estimating their structural model, they also found that implementing an approval voting system would reduce the distortions by 7.5\%. 
They also find that term limits may reduce distortion but increase corruption.

Most of the literature on pandering from public economics focuses intently on theoretical models to explain behavior, but comparatively little empirical testing of that behavior. 
This literature on pandering looks primarily at the incentives to pander, noting such studies as \cite{ASHWORTH2010838}, \cite{MASKIN201979},  and \cite{ENIKOLOPOV201474}. 
The Ashworth study focuses on the role of media in pandering and develops a game-theoretic model of policy selection when a media commentator is present, which declares a belief on states of the world. 
They define pandering as following voters' desires, even if their signal disagrees with what the voters want. 
From this setup, introducing a media commentator will decrease pandering for incumbents facing weak challengers but could increase pandering from incumbents with strong challengers. 
The Maskin study focuses instead on pandering, as defined as targeting spending towards interest groups to signal that politicians share their concerns. 
Enikolopov, on the other hand, is a more empirical study that shows that elected politicians are more prone to targeted redistribution efforts than appointed public officials and builds a model consistent with that concept by focusing on patronage jobs in local government. 
Enikolopov shows that the number of public employees is pro-cyclical with the election cycle, where the number of public employees increases in election years and decreases in off-election years. 
Finally, they show that older, non-elected officials increase hiring. 
They claim this is because younger non-elected officials have more substantial career concerns, which is profoundly similar to the model presented in the next section. 

However, this study borrows the most from the \cite{fowleretalquidproquo} study, which uses a combination of regression discontinuity and first-differences design to study whether there is evidence of successful corporate campaign contributions influencing the stock prices of the donors. 
From this two-pronged approach, Fowler et al. found that there really is no impact of a preferred candidate on winning, and thus it is hard to argue that campaign contributions are a profitable venture for companies. 
We will use a similar two-pronged approach to determine if newer politicians are more likely to spend conspicuously than experienced politicians. 