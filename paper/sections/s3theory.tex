\section*{The Fake-it-Till-You-Make-it Model and Rival Explanations}
To illustrate the electoral dynamics of the menu program, adjust the classic career-concerns model illustrated in \cite{gehlbach2021formal} to incorporate conspicuous expenditure and gain of competency through being in office. 
Most career concerns models focus on political effort and rent-seeking, \cite{barro_politiciancontrol} \cite{ferrejohn_acct}. 
However, this model focuses on the incentives to engage in publicly salient spending at the expense of voters' long-run interests. 
I call this model the ``fake-it-till-you-make-it'' model since it shows that conspicuous expenditures can be useful to attempt to ``mask'' a politician's competence until their competency and reputation allow them to succeed without the use of costly conspicuous expenditures.

The intuition of the model is similar to standard effort-based career concerns models. 
In the effort-based model, voters benefit from an elected official's competence and effort. 
After an election period, there is no incentive for the elected official to give any effort. 
Thus, when electing an official, voters only care about innate competence. The dilemma, however, is that voters only know how much utility they have gotten in the previous period where the incumbent was elected and thus must form beliefs about the competency of the elected official. 
Politicians, on the other hand, want to stay in office with as little effort as possible since effort is costly. 
In this case, instead of expending effort, an incumbent can expend effort by implementing projects that are publicly salient to voters and fund those projects at the expense of the community's longer-term interests. 
Imagine a politician choosing between a ribbon-cutting ceremony for a beautiful statue commemorating a major figure in the community or building bus lanes. 
It would be hard to imagine that the politician would not find the ribbon-cutting ceremony politically enticing. 
After all, it is free PR! In this case, such a tradeoff is costly since the incumbent needs to actively search for the correct project that will allow her to make this tradeoff. 
In the standard effort-based model, the equilibrium level of effort is proportional to the benefit of being in office times the probability density at the incumbent's level of competency. 
This proportionality is because as the probability density at that point increases, voters are more uncertain about the incumbent's competency, which means the incumbent needs to pander more. 
However, as experience pushes the politician towards the tails of the competency distribution, we would expect the probability density of the politician's competency to go to zero. 
Following the intuition of the standard effort-based model, we expect conspicuous expenditures to increase the more uncertain voters are about the expected confidence level and the higher the payoff of being in office. 
Furthermore, we would expect that unexpected increases in conspicuous expenditures increase the likelihood of getting reelected, much as how unexpected increases in effort could increase vote share. 

The model has the following actors: a mass of identical voters, an incumbent politician, and a challenger. 
This model has two periods: an initial election period where the incumbent and a challenger face off and a post-election period. 
During the election period, the incumbent chooses how much to spend conspicuously, knowing that their choices influence voters' decisions. 
Voters gain utility from the infrastructure stock ($K_t$), conspicuous expenditures ($S_t$), and competence ($\theta_t$). 
Politicians gain a wage of $w$ by being in office, but finding conspicuous projects is more costly than just investing in infrastructure with a cost term of $S_t^2/2$. 
We can denote the utility of the voter by:

\begin{align}
    u_t=\theta_t+\beta*S_t+\gamma*K_t
\end{align}

Voters observe their utility before voting but do not directly observe competence or conspicuous expenditures and must form beliefs about those values. 
Where $\theta_t$ is a random variable that represents the incumbent's competence when initially elected into office and has an expected value of 0, which comes from a cumulative distribution $F$; however, we allow for competence gained through holding office through the mechanism: $\theta_t=y_t+\theta_{t-1}$, where $y_t$ is a random variable indicating the competence gained from staying in office and has a positive mean $\alpha$. 
For this model, we will assume $\beta>\gamma$ to avoid a trivial solution of $S_t=0$. 
The infrastructure stock has the following law of motion:

\begin{align}
    K_t=(1-\delta)*K_{t-1}+i_t+\epsilon_t
\end{align}

In this equation, $\delta$ represents the rate of decay of the roads, and $i_t$ represents the infrastructure investment made. 
Finally, $\epsilon_t$ is a random shock to infrastructure that the politician sees but the voters only see indirectly and has an expected value of zero. 
Finally, note that $K_0=0$ and that there is no temporal discounting in this model for simplicity. 
Note that so long as $\gamma>0.5$ So, voters know their utility and form their beliefs about incumbent competence from their observations of the infrastructure stock. 
Thus their believed competence is the following in the first period:

\begin{align*}
    \tilde{\theta}_2&=\mathbb{E}(y_1)+u_1-\beta \tilde{S}_1-\gamma K_1 \\
    &=\alpha+u_1-\tilde{S}_1-\tilde{i_1}+\mathbb{E}(\epsilon_t)
\end{align*}

Note that the expectation must be correct to be a part of a rational equilibrium. 
However, from backward induction, we know both the challenger and the incumbent make the same decision to invest all the money in infrastructure. 
Thus, the voter's electoral decision only reflects the perceived competency of the incumbent versus that of a potential challenger. 
Implying that in the re-election period, the incumbent faces the following problem:

\begin{align}
    \max_{S_1} w*pr(\tilde{\theta}_2 \geq 0)-\frac{S_1^2}{2} \\
    \text{s.t} \ i_1+S_1=M
\end{align}

 However, the problem differs depending on whether this is the incumbents' first or second term in office. 
 If this is the incumbent's first term in office, voters know that the incumbent has only the advantage of experiencing a shock. 
 If this is the second term in office, voters will impute the incumbent's type with two experience shocks. 
 Let us start with the more straightforward first-term incumbent problem. 
 Knowing how voters formulate their beliefs and plugging in the budget constraint for $i_2$, the incumbents' problem can be rewritten as:

\begin{align}
        \max_{S_1} \left ( w*pr(\theta_1+\alpha+ (\beta-\gamma)(S_1-\tilde{S_1}) \geq 0 )-\frac{S_1^2}{2} \right)
\end{align}

Since $\theta_1$ comes from the distribution $F$, we can evaluate the probability, which results in the following equation:

\begin{align}
        \max_{S_1} w*(1-F((\beta-\gamma)(\tilde{S_1}-S_1)-\alpha))-\frac{S_1^2}{2}
\end{align}

Taking our first order condition and imposing rational expectations, we arrive at the optimal conspicuous expenditure of the second term challenger-winner, $S_{2c}^*$ below.

\begin{align}
    S_{1c}^*=f(-\alpha)*w
\end{align}

So, the victorious challenger's FOC is practically identical to the results derived in \cite{gehlbach2021formal}, where conspicuous expenditures increase with the office's value and the area's density near the cutoff, where the cutoff is lower due to experiential gain. 
By a similar process, we can conclude that the twice-experienced will have the following condition:

\begin{align}
    S_{1c}^*=f(-2\alpha)*w
\end{align}

How do conspicuous expenditures impact re-election odds, in any case? So long as voters anticipate the conspicuous expenditures, it does not. 
However, any tremendous amount of conspicuous expenditures would cause an increase in the re-electability of the incumbent. 
Because voters expect this behavior, the odds of re-election are $1-F(-2 \alpha)$ for the reelected incumbent and $1-F(-alpha)$ for the first-time incumbent. 
The model also would expect that reelected incumbents pander less than first-time incumbents and are more likely to be reelected. 
However, equation 6 does not necessarily show that conspicuous expenditures is lower under an experienced incumbent. 
That is only the case if the distribution is uni-modal at zero, such as under a normal distribution. 
If the distribution of competency were, say, uniform, there would be no effect. 
If it was bimodal, then the result could be reversed. 
The intuition of this result is that the lower the probability density of an incumbent's type, the more certain voters are about his type, so there is less incentive to pander. 
So, under a regime where voters value competence and incumbents gain competence over time, so long as the underlying distribution of competence, $F$, is uni-modal, we would expect that reelected incumbents would pander less than challengers who were elected and get reelected more as they push further and further out to the ``tails'' of the competency distribution. 
If we adjust the model to incorporate more experienced candidates, we would see that. 
 Thus, conspicuous expenditures is the ``fake-it'' element of the model, and the experience gained from being in office is the ``make-it'' element of the model. 

Nevertheless, this electoral effect is not the only way for these conspicuous expenditures to decrease over time. 
Another two mechanisms, which I dub the ``informational theory'' and the ``selection theory,'' can also explain many of the same results of the electoral incentive theory. 
The informational theory relies on experience as well but is not electorally motivated. 
This theory compares the knowledge of what the community needs from the public and the alderman. 
The community recommends a wide variety of projects, but it is easier to find projects that are conspicuous and easy to see than those that satisfy long-term interests. 
Inexperienced alderpersons who do not know the community's infrastructure needs may resort to conspicuous expenditures until they gain the knowledge necessary to make intelligent infrastructure allocations. 
The pure-informational theory contrasts with the electoral theory insofar that the electoral theory predicts no/low off-menu expenditures in non-campaigning periods and large off-menu expenditures in campaigning periods. 

The ``selection theory'' goes that voters select politicians more willing to listen to their needs, which are not well-correlated with what is available on the alder-manic infrastructure menu. 
Thus, in this theory, voters select politicians more receptive to their needs. 
However, politicians become more entrenched over time and less receptive to voter needs. 
Thus, the empirical project is to determine which of these ways of thinking about the program is more likely to be true. 
With the informational theory, we would not necessarily expect election outcomes positively correlate with experience. 
However, the voter selection theory would show that off-menu expenditures correlate with good election performance but would degrade as the incumbent becomes more politically entrenched, even in re-election periods, regardless of the candidate's electability. 

The theme from this section is that the common knowledge that politicians who are more established no longer need to expend much effort (either on conspicuous expenditures or other activities) to get reelected has a basis in a simple rational choice approach to experience in politics. 
A politician who is more experienced may not need to pander as much as long as the density of the left tail of the competency distribution decreases as competency decreases. 
This lessening occurs because they are more confident that the competency of the incumbent is enough to beat the challenger, and so the benefits of conspicuous expenditures are diminished. 