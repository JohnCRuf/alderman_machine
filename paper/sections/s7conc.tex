\section*{Conclusions}

This paper presents a theoretical argument that politicians may shift to publicly salient ``conspicuous'' expenditures to ``shore up'' their political position. 
Specifically, we test this hypothesis in the context of Chicago's Aldermanic Menu Program, which allows local politicians to tailor infrastructure expenditures in their wards unilaterally. 
We do this by showing that these politically salient expenditures, assumed to be ``off-menu'' expenditures, significantly impact an incumbent's vote-share through a simple discrete choice model, where TWFE removed some endogeneity. 
That analysis showed that off-menu expenditures influence voters. 
We then show with a regression discontinuity and diff-in-diff design that wards with new politicians spend a little more than \$100,000 more on politically salient expenditures than wards that re-elect incumbents or wards where incumbents do not retire. 
This argument is not without its flaws, however. 
The regression discontinuity analysis has a clear density discontinuity, and the underlying relationship between vote-share and off-menu expenditures is not continuous. 
The density discontinuity and placebo tests, respectively, substantiate these insights. 
However, the diff-in-diff argument presented seems more credible, as by inspection of the data and by a formal event-study parallel pre-trends test showing that, at the very least, the parallel trends assumption seems to work in the pre-period. 

Overall, the thesis shows some evidence for the claim that newer alderpersons are more likely to spend on off-menu expenditures. 
However, because the increase in off-menu expenditures takes place in off-election periods, it is unlikely that this is due to deliberate conspicuous expenditures for political advantage. 
In particular, if electoral motives induced this, then the result of the diff-in-diff that the primary effect on expenditures is in the period furthest away from the next election seems puzzling. 
Thus, the results of the diff-in-diff, if they provide evidence of anything, provide evidence more towards that of an experiential mechanism where alderpersons rely on public knowledge to allocate resources until they learn how to allocate it themselves or a selection mechanism where voters elect politicians most likely to listen to their input, which favors immediate interests. 
Furthermore, note that the evidence for any effect is weak at best due to the relatively high standard errors. 
More data is necessary to make more substantive conclusions on the nature of these expenditures, as it would ameliorate the estimation concerns with the diff-in-diff and allow a more precise estimation of the regression discontinuity analysis.