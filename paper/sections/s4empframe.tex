\section*{Empirical Framework}
This thesis employs three key empirical techniques: A discrete-choice model of voter choice, Differences-in-Differences estimation, and regression discontinuity. 
The rudimentary discrete-choice model will show that off-menu expenditures are an important consideration for voters and that voters are generally more likely to vote for incumbents who spend more on off-menu items. 
The regression discontinuity and the differences-in-differences analysis both estimate the effect of having a new alderman on conspicuous expenditures or off-menu expenditures. 
However, note that the theoretical model implies that losing politicians should be perceived as quite different from winning politicians, even near the cut-off. 
The fake-it-till-you-make-it model says that politicians who lose should be incompetent! Thus the regression discontinuity serves as a way to look for alternative hypotheses. 
For the regression discontinuity technique, I will exploit quasi-random variation near the electoral cut-off of 50\% to estimate the impact of a new alderman on the concentration of public expenditures and the percent of off-menu expenditures from the menu program. 
In my sample to create a small regression discontinuity study comparing wards that just barely elected incumbents vs. 
wards that just barely elected challengers. 

The discrete-choice approach only shows that off-menu expenditures matter for the voter choice problem. 
This finding reinforces the theoretical framework developed insofar that the problem faced by the voter is similar. 
Under the framework for this study, we will only consider off-menu expenditures in the discrete choice model. 
Voter i's utility of voting for candidate j is given by:

\begin{align*}
    V_{j}= \beta S_{j}+ \gamma_{j}+\nu_{t}+\epsilon_{j}
\end{align*}

$c$ indicates the choice of candidate: incumbent or challenger, and $S$ indicates the off-menu expenditures. 
$\epsilon$ is an idiosyncratic utility shock that each voter gets, and $\gamma_{j}$ is a fixed effect per ward. 
Finally, $\nu_{t}$ is a time-period fixed effect. 
 By assuming that $\epsilon$ takes the form of type-1 extreme value distribution and that 

\begin{align*}
    P(V=j|P)=\frac{exp( \beta S_{j,t} + \gamma_{j,t})+\epsilon_{i,j}}{exp(\beta S_c+\gamma_{j,t})+exp(\beta S_c+\gamma_c+\beta S_I+\gamma_{j,t})}
\end{align*}

The denominator is one because due to the run-off system Chicago employs, there are only challengers and incumbents to choose between. 
Normalizing the utility of voting for the challenger to be 0 always, we can use the inversion technique \cite{berry1994} and take the difference in log voting shares to estimate the impact of increased off-menu expenditures on the utility of voting for the incumbent. 
Furthermore, this allows us to bypass the problem of figuring out how much a challenger would have spent on off-menu items, as we only need data on how much the incumbent spent. 
Thus we arrive at the following estimation equation:

\begin{align*}
    \log \pi_{j,I}- \log \pi_{j,c}= \beta S_{i,j}+ \gamma_{j}+\nu_{t}+\epsilon_{i,j}
\end{align*}

Where the left-hand side represents the Paakes transformed version of market shares, and the left-hand side is our regression equation with $\gamma_{j,t}$. 
So long as the assumptions on the distribution of the error term are accurate: IE, they are exogenous (conditional on fixed effects), and if they are distributed according to a type-1 extreme value distribution, then our coefficient $\beta$ will tell us how much more the representative voter values a candidate who engages in more off-menu expenditure. 
This fact will allow us to incompletely test the model's implication that unexpected off-menu expenditures increase incumbent vote-share.


Moving on to the regression discontinuity empirical framework, with the regression discontinuity approach, we are aiming to show that, as predicted in the theoretical model, younger aldermen tend to spend more on politically salient off-menu expenditures. 
To do this, we will estimate the following equation:

\begin{align*}
    offmenu=\beta_0+\beta_1 IVS + \delta IW + \beta_2 IW\times IVS+u_i
\end{align*}

 There are two primary concerns with the regression discontinuity design. 
 $offmenu$ is the off-menu expenditures, $IVS$ is the incumbent's vote share, $IW$ is a dummy variable of whether the incumbent wins and $u_i$ is an error term. 
 The main worry with this analysis is if there is evidence of discontinuity at the cut-off, as this may indicate that there is a sorting mechanism that introduces a selection bias into our estimator. 
 In particular, this could bias the discontinuity estimate if some well-connected incumbents can gain endorsements to move from the left side to the right side of the cut-off and could make the estimates significantly larger if being well-connected is associated with lower off-menu expenditures. 
 However, a similar story that possibly wouldn't bias the estimate would be if aldermen to the left of the cut-off selectively retire from office, thus removing them from the sample. 
 The second concern is that the fundamental relationship between vote-share and off-menu expenditures is discontinuous. 
 This phenomenon can arise if, for example, a large number of candidates and aldermen will not go ``off-menu'' regardless of the electoral circumstance. 
 If this happens, the estimator will not be informative as it could simply measure the correlation between off-menu refusal and vote-share.


Due to the panel structure of the menu-allocation data, we can also take another approach to estimate the impact of a new alderman on menu allocations: differences in differences. 
For this, there is an array of options for DiD estimators. 
We will choose two: the group-time average treatment (ATT(g,t)) effect framework of \cite{CALLAWAY2021200} and the standard TWFE DiD estimator. 
While the standard two-way fixed effects estimator is likely valid due to a single treatment period and single treatment group, the ATT(g,t) framework also allows treatment anticipation, which may be especially important in the case of retiring aldermen. 
The simple TWFE approach is detailed below, where $treat_{w,t}$ is a dummy variable indicating the post-treatment period, and $\alpha_w, \alpha_y$ indicates the year and ward fixed effects, respectively. 

\begin{align*}
    offmenu=\beta_0 + \sum_{k=1}^4 \beta_k treat_{k,w,t} + \sum_w \alpha_w + \sum_y \alpha_y
\end{align*}

In this analysis, there are four post-treatment periods, with each treatment period getting a separate estimated effect $\beta_k$. 
If the theory developed is correct, we should see increased off-menu expenditures leading to the following election in the treatment group. 
The key potential issue in this framework is a violation of the parallel trends assumptions. 
Some avenues for this violation include due to anticipation, where aldermen change their allocative behavior in response to either a losing electoral position, an impending retirement, or both. 
Aldermen may put less effort into finding better ways to allocate the menu budget when they know they will leave office soon, so we may expect parallel trends not to hold if these effects dominate. 
Luckily, the lack of budgetary spillovers assists us insofar that we can be confident that spillover effects will not be an issue. 
Furthermore, we can directly test if the trends are parallel in the pre-period due to the wide panel structure of the data.