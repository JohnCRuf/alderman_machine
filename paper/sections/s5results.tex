\section{Results}

First we can look at the result of the competitive elections design.
We compute the dynamic aggregation of the average treatment effect on the treated group for the least and most supporting precincts and display the results in Table \ref{tab:att_comparison_close_election}.
Also shown are the estimate's standard errors, confidence intervals, and wald-test p-values for the pre-test of the parallel trends assumption.
For the close election design we see that both estimated ATTs are not statistically significant.
Furthermore, we see that the least supporting precinct's ATT is not particularly economically significant either. 
Half a percent of the menu program's budget is roughly \$7,500. 
That is not even enough to afford two street-wide speed bumps.
Additionally concerning are the extremely low p-values for the pre-trends test.
This indicates that the close election design likely does not guarantee that parallel trends hold.
This is may be because factors that influence infrastructure spending and needs are correlated with the the election outcome.
Alternatively, it could be because the margin chosen is too large to ensure that the two groups are similar.
The results of the competitive elections design is very sensitive to the number of precincts included, often changing the sign of both estimated ATTs.
\begin{table}[ht]
    \centering
    \caption{Comparison of Average Treatment Effects: Competitive Election Designs}
    \label{tab:att_comparison_close_election}
    \begin{tabular}{lcc}
    \hline
     & Least Supporting Precincts ATT & Most Supporting Precincts ATT \\
    \hline
    ATT & 0.5003 &  -1.5105 \\
    Std. Error & 0.4512 &  1.120 \\
    95\% Conf. Int. & (-0.384, 1.385) & (-3.705, 0.684) \\
    Pre-Trends P-value & 0.005  & 0.0755 \\
    Obs & 1680 & 1680 \\
    \hline
    \end{tabular}
\end{table}

We can look at how the average treatment effect changes over time in Figure\ref{fig:att_comparison_close_election_bottom} which depicts how the least supporting precincts estimated treatment varies over time.
As can be inferred from the standard errors in Table \ref{tab:att_comparison_close_election}, the estimated effect over time is incredibly noisy and unreliable.
This is likely because if you think about an optimal social planner, they would not allocate spending using a highly auto-correlated spending rule.
More likely, they would allocate spending based on the state of decay of the infrastructure stock of the precincts.
If you imagine that close elections push aldermen closer to a hypothetical social planner, then you would expect lots of noise in the estimated effect.
This noise is also present when we look the top supporting precincts.
\begin{figure}[ht]
    \centering
    \includegraphics[width=0.8\textwidth]{input/hte_figure_combined_support_bottom.png}
    \caption{Average treatment effect over time for least supporting precincts: competitive election design}
    \label{fig:att_comparison_close_election_bottom}
\end{figure}



Next we can look at the results of the indictment design in Table \ref{tab:att_comparison_corruption}.
We note that the treatment effect for the least supporting precincts is over five time larger than the treatment effect estimated in the competitive election design.
We also see a much larger wald-test p-value for the pre-trends test.
This indicates that perhaps the indictment design's assumption is more believable than the close election design's assumption.
However, we also see that the most supporting precinct's treatment effect, while statistically significant is similar in magnitude to the competitive election design's treatment effect.
The results shown Table~\ref{tab:att_comparison_corruption} are only sensitive to increasing the number of precincts, not decreasing it. 


\begin{table}[H]
    \centering
    \caption{Comparison of Average Treatment Effects: Indictment Designs}
    \label{tab:att_comparison_corruption}
    \begin{tabular}{lcc}
    \hline
     & Least Supporting Precincts ATT & Most Supporting Precincts ATT \\
    \hline
    ATT & 2.5881 & -1.1467 \\
    Std. Error & 0.7776 & 0.3191 \\
    95\% Conf. Int. & (1.064, 4.1121) & (-1.7721, -0.5214) \\
    Pre-Trends P-value & 0.1989 & 0.17374 \\
    Obs & 1144 & 1144 \\
    \hline
    \end{tabular}
\end{table}

Finally we can look at how the average treatment effect changes over time in Figure\ref{fig:att_comparison_corruption_bottom} which depicts how the least supporting precincts estimated treatment varies over time.
We see that the standard errors, while large, are much smaller and the mean estimate is much more stable.
However, we see some evidence of a pre-trends violation in the year before the first treatment, as the 2019 estimate is significantly smaller than the other estimates.
This holds for the most supporting precincts as well, albeit in reverse.
Thus it seems that as the incumbent alderman is leaving office, they are allocating more spending to their most supporting precincts from their least supporting precincts.
Overall, these results must be taken with a grain of salt, due to the high standard errors the likely pre-trends violation.

\begin{figure}[H]
    \centering
    \includegraphics[width=0.8\textwidth]{input/hte_figure_corruption_bottom_8_precincts.png}
    \caption{Average Treatment Effect over Time: Indictment Designs}
    \label{fig:att_comparison_corruption_bottom}
\end{figure}

