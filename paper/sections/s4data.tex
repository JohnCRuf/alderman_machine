\section{Data Description}\label{sec:data_description}
This paper uses a panel of located menu expenditures containing the yearly allotted allocations
from aldermen and their respective locations from 2005 to 2022. 
This dataset comes from annual menu spending reports publicly available from 2011 through 2022. 
OBM stores these reports in PDF format.
They range from 150 to 500 pages long and contain information on each project's cost, description, and location. 
For example, the alderman reports a pair of intersections in a simple road resurfacing. 
In an alley resurfacing, the alderman often reports the four surrounding intersections.
The location and description information is up to the alderman's discretion, and the quality of
the information varies widely. 
I obtained data before then from records that were not previously publicly available through a FOIA request to the OBM \citep{OBM_datasource}.  
I scraped each of these reports and cleaned the resulting cost total, ward, and location description data.
I then used the location description text to locate each project's described vertices using the Census geocoding API.
If the Census's API failed, I automatically used Google Maps' API instead.
In total, 43,596 projects needed to be located.  
I successfully located 83\% of them.
For example, spending on playground equipment would be a singular point, while spending on a street would be a line, and spending on all alleys within a given block would be a polygon.
The spending for each project was then assigned to the precincts that overlapped with it by area overlapped.
If a project were a \$10,000 quadrangle of $500 m^2$, where 60\% of the area was in precinct A, and 40\% was in precinct B, I would assign \$6,000 to precinct A and \$4,000 to precinct B.
I treated lines similarly, only using length instead of area.
This dataset contains 41,381 precinct-year spending observations.

Figure~\ref{fig:spending_hist} depicts two side-by-side histograms of the distribution of spending per precinct aggregated across the 2005-2011 period, which used the 2003-2011 ward boundaries and the 2012-2022 period with the 2012-2022 ward boundaries.
The decentralized nature of the menu program leads to a considerable variation in spending per precinct, but the distribution has a long right tail.
Both figures are winsorized at the 99th percentile to remove outliers.

\begin{figure}[H]
    \centering
    % First subfigure
    \begin{subfigure}[b]{0.45\textwidth} % [b] aligns at the bottom
      \includegraphics[width=\textwidth]{input/spending_histogram_2005_2011.png}
      \caption{2005-2011}
      \label{fig:sub1}
    \end{subfigure}
    \hfill % This adds some space between the two subfigures
    % Second subfigure
    \begin{subfigure}[b]{0.45\textwidth}
      \includegraphics[width=\textwidth]{input/spending_histogram_2012_2022.png}
      \caption{2012-2022}
      \label{fig:sub2}
    \end{subfigure}
  
    \caption{Distribution of Spending per Precinct for both ward maps in the dataset}
    \label{fig:spending_hist}
  \end{figure}

Next, Figure~\ref{fig:spending_map} depicts within-ward spending variation across Chicago using the 2012-2022 precinct map. 
The map shows that spending is often concentrated in a few precincts.

\begin{figure}[H]
    \centering
    \includegraphics[width=0.65\textwidth]{input/whole_chicago_map_2012_2022.png}
    \caption{Map of Spending per Precinct, 2012-2022}
    \label{fig:spending_map}
\end{figure}

To avoid issues with different levels of ``observable'' spending from year to year, I use the fraction of spending located in a precinct in a given year as the dependent variable for subsequent analyses.
``Observed'' in this context means one of the two methods above successfully located it to a location in Chicago.
For example, if a precinct received \$50,000 in spending in 2019, but only \$900,000 of its ward's \$1.5M budget was located, then the fraction of spending located in that precinct in 2019  would be $\frac{50,000}{900,000}*100 =5.6\%$.
I use this instead of dollar amounts because the amount of observable spending can shift from year to year, but that doesn't mean that alderman is prioritizing the precinct any less that year.
Thus, if $Y_{py}$ is the fraction of spending located in precinct $p$ in year $y$, then $Y_{py} = \frac{S_{py}}{S_{wy}}$, where $S_{py}$ is the spending in precinct $p$ in year $y$, and $S_{wy}$ is the total amount of ward $w$'s successfully located spending in year $y$.
Table~\ref{summary_stats} depicts summary statistics for this variable.
Due to the long right tail of the distribution, the mean is much larger than the median.
Half of the precincts get almost no spending each year.

\begin{table}[H]
\caption{Summary statistics of fraction of total spending by precinct from 2012 to 2022}\label{summary_stats}
\centering
\begin{tabular}[t]{ccccc}
\toprule
mean & median & sd & upper quartile & lower quartile\\
\midrule
2.42 & 0.23 & 4.48 & 3.35 & 0\\
\bottomrule
\end{tabular}
\end{table}