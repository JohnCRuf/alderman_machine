\section{Methods}
\begin{frame}{Differences-in-Differences strategy}
    I employ a differences-in-differences identification strategy. 
    The simplest formation of this model is given by:

    \begin{equation}\label{eq:standard_did}
        Y_{pt} = \alpha_{p} + \gamma_{t} + \beta T_{pt} + \epsilon{pt}
    \end{equation}
    Where $Y_{pt}$ is the fraction of observed spending in precinct $p$ in year $t$. $T_{pt}$ is a dummy variable that is equal to 1 if the incumbent alderman was removed by office in year $t$. $\alpha_{p}$ and $\gamma_{t}$ are precinct and year fixed effects respectively. $\epsilon_{pt}$ is the error term.

    Of the new DiD estimators, I use the heterogeneous-treatment effect robust estimator proposed by (Callaway et al. 2021).
\end{frame}

\begin{frame}{Parallel Trends Justification}
    There are two subsets of aldermen that I focus on
    \begin{enumerate}
        \item Wards with highly competitive elections (within 10\% margin)
        \begin{itemize}
            \item Close elections justify parallel trends as the incumbent is equally likely to win as the challenger, thus incumbents who win and incumbents who lose should have similar expectations ex ante.
        \end{itemize}
        \item Wards where the incumbent was indicted for corruption
        \begin{itemize}
            \item Indictments justify parallel pre-trends due to the inherently unknown nature of federal corruption and racketeering investigations. Also, indicted aldermen typically do not face serious electoral challenges.
            \item Comparing indicted aldermen to aldermen who also do not face serious challenges justifies the parallel trends assumption as ex ante they both should have similar and stable spending preferences.
        \end{itemize}
    \end{enumerate}
\end{frame}